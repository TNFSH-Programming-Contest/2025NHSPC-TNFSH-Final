\PassOptionsToPackage{dvipsnames}{xcolor}
\documentclass[hyperref,UTF8,notheorems,xcolor={dvipsnames}]{beamer}
\usepackage[dvipsnames]{xcolor}
\usetheme{Rochester}
\usepackage{amsmath, mathtools, graphicx}
\usepackage{standalone}
\usepackage[most]{tcolorbox}
\usepackage[normalem]{ulem}
\usepackage{xeCJK}
\usepackage{fontspec}
\usepackage{tikz}
\usepackage{pgfplots}
\usepackage{pifont}
\usepackage{fontawesome5}
\usepgfplotslibrary{polar}
\usepgflibrary{shapes.geometric}
\usetikzlibrary{calc,angles,quotes}
\defaultfontfeatures{Mapping=tex-text}
\usefonttheme{professionalfonts}
\usepackage{concmath}
\usepackage{minted}

\newcommand{\cmark}{\ding{51}}%
\newcommand{\xmark}{\ding{55}}%

\newcommand{\CC}[1]{#1}

\newcommand{\Cincluded}{{\small\cmark}}
\newcommand{\Cdefine}{{\small\cmark\faFile*[regular]}}
\newcommand{\Ccodeonly}{{\small\cmark\faFile*~}}
\newcommand{\Cnofocus}{{\small\faQuestion}}
\newcommand{\Cexmaybe}{{\small\xmark\faQuestionCircle}}
\newcommand{\Cexcluded}{{\small\xmark}}

\newcommand{\Iincluded}{\item[\hbox to 1.8em{\Cincluded\hfill}]}
\newcommand{\Idefine}{\item[\hbox to 1.8em{\Cdefine\hfill}]}
\newcommand{\Icodeonly}{\item[\hbox to 1.8em{\Ccodeonly\hfill}]}
\newcommand{\Inofocus}{\item[\hbox to 1.8em{\Cnofocus\hfill}]}
\newcommand{\Iexmaybe}{\item[\hbox to 1.8em{\Cexmaybe\hfill}]}
\newcommand{\Iexcluded}{\item[\hbox to 1.8em{\Cexcluded\hfill}]}

\DeclareRobustCommand{\rddots}{\text{\reflectbox{$\ddots$}}}


\usepackage{graphicx}
\graphicspath{ {./../images/} }

\def\codesize{\fontsize{8}{9}\selectfont}
\setmonofont[Mapping=]{Source Code Pro}
\setminted{fontsize=\codesize, linenos, frame=lines, mathescape, autogobble, tabsize=4}
\setCJKmainfont[AutoFakeSlant,BoldFont=Noto Sans CJK TC Bold]{Noto Sans CJK TC}

\setlength{\parskip}{\baselineskip} 
\newcommand{\btitle}[1]{{\secname} -- #1}

\theoremstyle{definition}
\newtheorem{theorem}{定理}
\newtheorem{lemma}{引理}
\newtheorem{property}{性質}
\newtheorem{corollary}{推論}
\newtheorem{problem}{例題}


\newtheorem{definition}{定義}
\AtBeginEnvironment{definition}{
  \setbeamercolor{block title}{fg=white,bg=red!70!black}
  \setbeamercolor{block body}{fg=black, bg=block title.bg!10!bg}
}

\newtheorem{exercise}{習題}
\AtBeginEnvironment{exercise}{
  \setbeamercolor{block title}{fg=white,bg=green!30!black}
  \setbeamercolor{block body}{fg=black, bg=block title.bg!10!bg}
}


\AtBeginSection[]{
%   \begin{frame}
%     \tableofcontents[currentsection,hideallsubsections]
%   \end{frame}
  \begin{frame}
  \vfill
  \centering
  \begin{beamercolorbox}[sep=6pt,center,shadow=true,rounded=true]{title}
    \usebeamerfont{title}\LARGE\insertsectionhead\par%
  \end{beamercolorbox}
  \vfill
  \end{frame}
}


\AtBeginSubsection[]{
  \begin{frame}
    \tableofcontents[subsectionstyle=show/shaded/hide]
  \end{frame}
}

\usepackage{ctable}
\usepackage{tabularx}

\setlength{\parskip}{\baselineskip}


\title{114 臺南一中學科能力競賽校內複選}

\hypersetup{CJKbookmarks=true}
\begin{document}

\author{題解}
\date{Sep 25 2025}

\begin{frame}
	\titlepage
\end{frame}

\section*{Overview}

\begin{frame}[fragile]{\btitle{預期解出人數}}
	預測第五線 : \textcolor{blue}{400}
	\begin{center}
		\begin{tikzpicture}
			\begin{axis} [%
				ybar,
				bar width=12pt,
				xmin=0.5,
				xmax=6.5,
				ymin=0,
				xtick={1,2,3,4,5,6},
				xticklabels={A,B,C,D,E,F}]
			\addplot plot coordinates {
				(1,10) 
				(2,1) 
				(3,5)
				(4,4)
				(5,8)
				(6,2)
			};
			
			\addplot [color=white] plot coordinates {
				(1,1) 
				(2,1) 
				(3,1) 
				(4,1)
				(5,1)
				(6,1)
			};
			\end{axis}
		\end{tikzpicture}
	\end{center}
\end{frame}

\begin{frame}[fragile]{\btitle{實際解出人數}}
	預測第五線 : \textcolor{blue}{400}  實際第五線 : \textcolor{red}{357}
	\begin{center}
		\begin{tikzpicture}
			\begin{axis} [%
				ybar,
				bar width=12pt,
				xmin=0.5,
				xmax=6.5,
				ymin=0,
				xtick={1,2,3,4,5,6},
				xticklabels={A,B,C,D,E,F}]
			\addplot plot coordinates {
				(1,10) 
				(2,1) 
				(3,5)
				(4,4)
				(5,8)
				(6,2)
			};
			\addplot plot coordinates {
				(1,8) 
				(2,1) 
				(3,4) 
				(4,6)
				(5,3)
				(6,2)
			};
			\end{axis}
		\end{tikzpicture}
	\end{center}
\end{frame}

\section{出題者想說的 和 Fun Fact}

\begin{frame}[fragile]{\btitle{前言}}

	出題出很久,希望大家都有好好打 
	\pause


	除了 E 外預測的算滿準的,我以為大家數學都很好
	\pause

	題目應該比去年簡單一點

\end{frame}

\begin{frame}[fragile]{\btitle{pB}}
	\begin{itemize}
		\item 本來以為可以用 AI666 那題的 greedy 作法,寫完後才發現假解。
		\pause
		\item 花了 2 小時和暴力 DP 跑過對拍,官解應該沒有假解。
	\end{itemize}
\end{frame}

\begin{frame}[fragile]{\btitle{pE}}
	\begin{itemize}
		\item Hikaru 是目前西洋棋等級分第二高的選手,他一次變出五隻城堡的影片 https://www.youtube.com/watch?v=JW3FOQWzlvA
		\item 對手其實是印度人,為了湊出中國剩餘定理才讓他變中國人,但這題應該跟中國剩餘定理完全沒關係。

	\end{itemize}
\end{frame}


\begin{frame}[fragile]{\secname}
	\begin{enumerate}
		\item A : 構造題
		\item B : 單調隊列
		\item C : 貪心
		\item D : 互動題
		\item E : 排列組合
		\item F : 樹論
	\end{enumerate}
\end{frame}

\section{A. 部落衝突 (tribe)}


\begin{frame}[fragile]{\btitle{題目敘述}}
	給你 $N$ 個點,每個點都有一個顏色 $C_i$,問你能不能把這 $N$ 個點連成一棵樹且恰好 $K$ 條邊的兩端是不同顏色。
\end{frame}

\begin{frame}[fragile]{\btitle{子任務}}
	\begin{enumerate}
		\item $K = 0$
		\item $K = N - 1$
		\item $C_i \in \{1, 2\}$
		\item 無額外限制
	\end{enumerate}
\end{frame}

\begin{frame}[fragile]{\btitle{$K = 0$}}
	很顯然所有點都要是相同顏色才能達成。 

	因為要是有不同顏色的兩個點,他們之間一定有至少一條邊連接不同顏色。
\end{frame}

\begin{frame}[fragile]{\btitle{$K = N - 1$}}
	如果所有點都相同顏色就不可能達成。 

	否則我們可以選兩個不同顏色的點當代表,對於每個點都連到顏色不一樣的點上面就好。
\end{frame}

\begin{frame}[fragile]{\btitle{$C_i \in \{1, 2\}$}}
	綜合 subtask 1, 2,我們可以推論出如果只有一種顏色,只有 $K = 0$ 才能滿足。
	\pause

	如果兩種顏色都有,那只要 $K \ge 1$ 都能達成,我們可以用類似 subtask 2 的方式構造。 \\
	先選兩個不同顏色的代表點連起來,接著如果目前還沒有 $K$ 條相異顏色的邊,那我們就把點連到相異顏色的代表點,如果已經有 $K$ 條了,那我們就把點連到相同顏色的代表點就好。
\end{frame}

\begin{frame}[fragile]{\btitle{無額外限制}}
	延續 subtask 3,我們把所有顏色都拿一個點起來當代表。 
	\pause

	假設有 $c$ 種不同顏色,那 $K \ge c - 1$ 才會有解,因為我們一定要用 $c - 1$ 條不同顏色的邊將這 $c$ 種顏色連通。
	\pause

	接著用 subtask 3 的構造方法,如果目前還沒有 $K$ 條相異顏色的邊,那我們就把點連到相異顏色的點,如果已經有 $K$ 條了,那我們就把點連到相同顏色的點就好。
\end{frame}

\section{B. 切蛋糕 2(cake2)}

\begin{frame}[fragile]{\btitle{題目敘述}}
	我們定義一個合法的切蛋糕方式為一系列的區間 $(l_1, r_1), (l_2, r_2), ..., (l_k, r_k)$
	滿足 $l_1 = 1$, $r_k = n$,並且對於所有 $1 \le i \le k$ 都有 $l_i \le r_i$。
	且對於所有 $1 \le i < k$,都有 $r_i + 1 =  l_{i + 1}$。
	而這個切蛋糕方式所得到的好感值總和為 
	$$
	\sum_{t = 1}^{k} \begin{cases}
	0 & \text{if  } r_t - l_t + 1 < K \\
	\min \limits_{l_t \le i \le r_t} C_i & \text{otherwise  }
	\end{cases}
	$$
	我們要找出所有合法的切蛋糕方式中好感值總和最大的是多少。
\end{frame}

\begin{frame}[fragile]{\btitle{子任務}}
	\begin{enumerate}
		\item $N \le 3000$
		\item $K > 1$
		\item $|C_i| \le 1$
		\item 無額外限制
	\end{enumerate}
\end{frame}

\begin{frame}[fragile]{\btitle{$N \le 3000$}}
	如果你有寫出初選的 pD,那你應該要會這個子任務
	\pause

	定義 $dp[i]$ 為 : 只看 $1$ 到 $i$,分數總和最大會是多少。  
	因此有轉移式 $$dp[i] = \max(dp[i - 1], \max_{1 \le j < i}{(dp[j] + cost(i, j))})$$
	其中 $cost(i, j)$ 為選區間 $i$ 到 $j$ 的貢獻

	而最後 $dp[n]$ 即為答案
\end{frame}

\begin{frame}[fragile]{\btitle{$K > 1$}}
	可以想到一個貪心的結論,一定有一種最佳解切的區間長度都為 $K$ 或 $1$
	\pause

	因為是取 min 的關係,大於 $K$ 的區間都可以把他切成大小為 $K$ 的加上一些大小為 $1$ 的區間,使得整體貢獻不變。

	因此子任務 1 的轉移式就可以用單調對列之類的資料結構 $O(n)$ 轉移了
\end{frame}

\begin{frame}[fragile]{\btitle{$|C_i| \le 1$}}
	如果 $K > 1$,就用子任務 $2$ 的方式做就好。 \\
	下面都討論 $K = 1$ 的 case。
	\pause

	\begin{itemize}
		\item 如果區間大小大於 $1$,那這個區間的兩端一定都是負數。 \\
	\end{itemize}
	\pause

	顯然,如果兩端有正的,那把那些正的自己拿出來成為一個區間貢獻更大。
	 


	
\end{frame}

\begin{frame}[fragile]{\btitle{$|C_i| \le 1$}}
	因此我們考慮以下做法
	
	如果有連續的 $-1$,那可以貪心的把這些放在同一個區間合併成只有一個 $-1$。
	\pause

	如果有連續 $k$ 個 $1$,我們可以各自一個區間,由於區間兩端不會是正數,可以直接把他們視為一個 $k$。
	\pause 

	而 $0$ 可以選要自己一個區間還是跟 $-1$ 的區間合併,兩者都不影響那個區間的貢獻,因此可以直接把 $0$ 給刪掉。
	
\end{frame}

\begin{frame}[fragile]{\btitle{$|C_i| \le 1$}}
	如此一來,整個序列會變成 $a_1, -1, a_2, -1, a_3, ...$ 或 $-1, a_1, -1, a_2, -1, a_3, ...$,的形式,其中 $a_i > 0$,而這樣切成的區間就會是答案。
	\pause

	因為如果還需要將一些區間合併。假設還要將 $-1, a_l, -1, a_{l + 1}, ..., -1, a_r, -1$ 給合併。
	因為 $a_i > 0$,原本的總和一定是一個大於等於 $-1$ 的數,你合併後他的貢獻會變成 $-1$,只會變得更差而已,所以不會有這種狀況發生。

\end{frame}

\begin{frame}[fragile]{\btitle{無額外限制}}
	我們用 subtask 3 說到的方法,把一群負的合併成一個最小值,一群正的加起來。
	我們會得到一個 $a_1, b_1, a_2, b_2, ...$ 或 $b_1, a_1, b_2, a_2,...$ 的序列,其中 $a_i > 0, b_i < 0$
	\pause

	我們假設 $a_1, b_1, ..., a_{i - 1}, b_{i - 1}$ 這樣的切法是最好的
	\pause

	顯然,再放一個正的在後面也會是最好的,所以 $a_1, b_1, ..., a_{i - 1}, b_{i - 1}, a_i$ 也會是最好的。
\end{frame}

\begin{frame}[fragile]{\btitle{無額外限制}}
	$a_1, b_1, ..., a_{i - 1}, b_{i - 1}, a_i$ 我們要放一個負的 $b_i$ 在後面
	\pause
	\begin{itemize}
		\item 如果 $b_{i - 1} > b_i$
		\pause
			\begin{itemize}
				\item 如果 $b_{i - 1} + a_{i} \le 0$ \\
				我們可以直接把 $b_{i - 1}, a_i$ 跟 $b_i$ 合併在一起並遞迴下去,這樣會讓代價從 $... + b_{i - 1} + a_{i} + b_{i}$ 變為 $... + b_i$,不虧。
				\pause

				\item[$\rightarrow$] 像是 -8, 5, -4, 3 要放 -6 進去,可以變成 -8, 5 放 -6 進去
				\pause
				\item 否則 $b_{i - 1} + a_{i} > 0$ \\
				就類似上面區間兩端不會有正數那樣,後面的數如果要合併要嘛和 $b_{i}$ 合併,要嘛和更前面 $< b_{i}$ 的那個合併,不可能有區間斷在 $b_{i - 1}$ 左邊了。 \\
				因此我們可以直接把 $a_{i - 1}$ 看成 $a_{i - 1} + b_{i - 1} + a_{i}$ 並遞迴下去。
				\pause

				\item[$\rightarrow$] 像是 -8, 5, -4, 9 要放 -6 進去,可以變成 -8, 10 放 -6 進去

			\end{itemize}

	\end{itemize}
	
\end{frame}

\begin{frame}[fragile]{\btitle{無額外限制}}
	\begin{itemize}
		\item 否則 $b_{i - 1} < b_i$ 
			\pause
			\begin{itemize}
				\item 如果 $a_{i} + b_{i} \le 0$ \\
				我們可以直接把 $a_i, b_i$ 跟 $b_{i - 1}$ 合併在一起,這樣會讓代價從 $... + b_{i - 1} + a_{i} + b_{i}$ 變為 $... + b_{i - 1}$,不虧。
				\pause
				\item[$\rightarrow$] 像是 -8, 5 要放 -6 進去,那就變成 -8 然後不放東西
				\pause
				\item 否則 $a_{i} + b_{i} > 0$ \\
				我們直接把 $b_i$ 丟到後面就好
				\pause

				\item[$\rightarrow$] 像是 -8, 5 要放 -3 進去,那就直接放進去變 -8, 5, -3

			\end{itemize}

	\end{itemize}
	\pause

	這樣一直做下去就會是好的,至於更嚴謹的證明這裡篇幅不夠就留給各位讀者回去當練習了。
	
\end{frame}

\section{C.  手錶 (clock)}

\begin{frame}[fragile]{\btitle{題目敘述}}

	給 $N$ 個 $(A_i, B_i)$ 的 pair 和 $M$,你每次操作可以從這 $N$ 個 pair 中選擇一個集合和一個整數 $k$,
	並將這個集合裡面的 $A_i$ 變為 $(A_i + k) \mod M$,每個 pair 最多只能被選到一次,問你最少要幾次操作才能讓所有 $A_i$ 大於等於 $B_i$。

\end{frame}

\begin{frame}[fragile]{\btitle{子任務}}
	\begin{enumerate}
		\item $A_i$ 皆相同
		\item $N \le 2000$
		\item $B_i$ 皆相同
		\item 無額外限制
		\item[$\triangle$] 每個子任務中,如果並非最少次調整,但滿足輸出格式,可以獲得總分乘以 0.3 的分數。
	\end{enumerate}
\end{frame}

\begin{frame}[fragile]{\btitle{30 分}}
	
	我們可以直接花 $N$ 次調整,每次都把 $A_i$ 調到 $B_i$ 就好
	\pause

	根本來送分的,希望大家都有拿到
	
\end{frame}

\begin{frame}[fragile]{\btitle{$A_i$ 皆相同}}
	很顯然如果所有 $A_i$ 一開始都大於等於 $B_i$ 就不用調整。
	\pause

	如果要調整,那我們將所有 $A_i$ 一起變為 $M - 1$ 就一定符合條件。
\end{frame}

\begin{frame}[fragile]{\btitle{$N \le 2000$}}
	可能下方 greedy 可以 $O(n^2)$ 做,沒想到什麼特別的解。

\end{frame}

\begin{frame}[fragile]{\btitle{$B_i$ 皆相同}}
	可能某些 greedy 的方法會對,沒想到甚麼特別的解。
	
\end{frame}

\begin{frame}[fragile]{\btitle{無額外限制}}
	不考慮那些 $A_i$ 一開始就大於等於 $B_i$ 的。

	列一下式子可以發現,每個可以選擇的 $k$ 的範圍為 $B_i - A_i \le k < M - A_i$。 
	\pause 

	因此我們可以轉換一下題目,給你 $n$ 個線段,每個線段為 $[B_i - A_i, M - A_i)$。
	你要選盡可能少的點,使得每個線段都被至少一個點覆蓋到。
\end{frame}

\begin{frame}[fragile]{\btitle{無額外限制}}
	這個問題的答案會和選最多不相交的線段 (CSES Movie Festival 那題) 一樣。

	至於為什麼這裡篇幅不夠就留給各位看完這題的解法後回家自己想想看了
\end{frame}

\begin{frame}[fragile]{\btitle{無額外限制}}
	我們可以對每條線段按右界排序的順序去看,一開始維護一個集合 $S$。  \\
	如果 $S$ 為空,那我們就把目前的線段丟進去 $S$。  \\ \pause
	假設 $S$ 裡右界最小的線段叫 $A$,我們現在看的線段叫做 $B$。  \\ 
	如果 $B$ 的左界和 $A$ 的右界有相交,那我們就把這個線段丟進 $S$ 裡。  \\ \pause
	否則, $A$ 和 $B$ 一定不能在同一次選到,而因為 $S$ 裡所有線段都和 $A$ 的右界有相交,因此可以選那個右界來把 $S$ 裡的線段都一起處理掉,因此把 $S$ 清空並把 $B$ 放進去。  \\ 
	大概證明是如果之後有一個線段 $X$ 能和 $A$ 這次一起處理掉,那線段 $X$ 也能和 $B$ 一起處理掉。至於更嚴謹的證明這裡篇幅不夠就留給各位讀者回去當練習了。
\end{frame}



\section{D. 隱藏的排列 2 (permutation2)}

\begin{frame}[fragile]{\btitle{題目敘述}}
	互動題,有一個 $1$ 到 $n$ 的隱藏排列 $p$。
	每次你可以給兩個數字 $l, r$,詢問有多少數對 $(i, j)$ 滿足 $l \le i < j \le r$ 且 $p_i > p_j$ (以下稱為 $l, r$ 內的逆序數對數量)。
\end{frame}

\begin{frame}[fragile]{\btitle{子任務}}
	\begin{enumerate}
		\item $n = 3$
		\item 無額外限制
		\item[$\triangle$] 觀察一下連續給分的公式,如果我們詢問所有合法的 $l, r$,那我們可以拿到 59.59 分。
	\end{enumerate}
\end{frame}

\begin{frame}[fragile]{\btitle{$n = 3$}}
	跟初選那題 $n = 3$ 的做法一樣,因為就 6 種可能的排列,我們觀察一下每個排列的問 $(l, r) =  (1, 2), (2, 3), (1, 3)$ 分別會是多少  
	\pause

	1 2 3 -> (0, 0, 0)  \\
	1 3 2 -> (0, 1, 1)  \\
	2 1 3 -> (1, 0, 1)  \\
	2 3 1 -> (0, 1, 2)  \\
	3 1 2 -> (1, 0, 2)  \\
	3 2 1 -> (1, 1, 3)  \\
	可以發現如果問 (1, 2) 和 (1, 3),這些排列對應的值兩兩相異,因此我們可以用這兩個詢問找到對應的那個排列。
\end{frame}

\begin{frame}[fragile]{\btitle{59.59 分}}
	知道了所有區間的逆序數對關係,那稍微排容一下就可以知道所有兩個數間的大小關係了。
	
	之後可以使用 std::sort 之類的方法來還原出隱藏的序列。

\end{frame}

\begin{frame}[fragile]{\btitle{無額外限制}}
	觀察一下分數,要拿到滿分最多只能問 $n - 1$ 次,並且根據 subtask 1,大概可以通靈出是問所有 $[1, i]$ 區間。
	\pause

	我們可以從 $p_1$ 看到 $p_n$,在看到第 $p_i$ 時,只需要維護好 $p_1$ 到 $p_i$ 之間的大小關係就好。
	也就是在不改變相對關係的情況下我們把 $p_1$ 到 $p_i$ 重新編號成 $1$ 到 $i$。 
	\pause

	假設我們知道這東西了,那把 $[1, i + 1]$ 的逆序數對數量減去 $[1, i]$ 的逆序數對數量。
	就相當於 $[1, i]$ 中有幾個數大於 $p_{i + 1}$,假設叫做 $k$,那只要把原先維護好的 $i - k + 1$ 到 $i$ 全部 + 1,並將 $p_{i + 1}$ 設為 $i - k + 1$ 就能維護好了。
	
\end{frame}


\section{E. 數城堡 (rook)}

\begin{frame}[fragile]{\btitle{題目敘述}}
	有一個 $N \times N$ 的棋盤和 $K$ 種顏色的的城堡,第 $i$ 種顏色的城堡有 $C_i$ 個。要算有多少種方法能把這些城堡全部放在這個棋盤上,使得任兩個城堡都不會互相攻擊到。
\end{frame}

\begin{frame}[fragile]{\btitle{子任務}}
	\begin{enumerate}
		\item $K = N$
		\item $M$ 是質數
		\item $K = 1$
		\item 無額外限制
	\end{enumerate}
\end{frame}

\begin{frame}[fragile]{\btitle{$K = N$}}

	假設 $s = \sum C_i$,由於 $C_i \ge 1$,因此這個子任務中 $s \ge N$。
	\pause

	因為每一排只能放一個城堡,所以最多只能放 $N$ 個城堡,所以 $s > N$ 答案就是 $0$。
	\pause

	$s = N$ 時先假設城堡顏色都一樣,那會有 $s!$ 種放法。
	
	在將城堡上色,有 $s!$ 種上色方式。

	因此答案會是 $(s!)^2$
\end{frame}

\begin{frame}[fragile]{\btitle{$K = 1$}}

	我們先選要放在哪 $s$ 排,哪 $s$ 列,所以會有 $\binom{N}{s}^2$ 種選法。
	\pause

	再將這 $s$ 個城堡放上去,會有 $s!$ 種放法,所以答案會是 $\binom{N}{s}^2 \times s!$
	\pause

	而 $\binom{N}{s} = \dfrac{N!}{s!(N-s)!}$ 我們可以數每個質數被乘到幾次後先約分再把他們給乘起來。

\end{frame}

\begin{frame}[fragile]{\btitle{$M$ 是質數}}
	延續 $K = 1$ 的子任務,我們再來把城堡上色,會有 $\dfrac{s!}{\prod{C_i!}}$ 種上色方法,所以答案會是 $\binom{N}{s}^2 \times s! \times \dfrac{s!}{\prod{C_i!}}$。
	
	因為 $M$ 是質數,可以用模逆元來算。
\end{frame}

\begin{frame}[fragile]{\btitle{無額外限制}}
	$M$ 是質數的做法,但先數每個質數被乘到幾次後先約分再把他們給乘起來。
\end{frame}


\section{F. 部落衝突 2 (tribe2)}

\begin{frame}[fragile]{\btitle{題目敘述}}
	給你一棵樹,每個點都有一種顏色,有 $Q$ 筆詢問。
	每次詢問問你從 $A$ 到 $B$ 的路徑上第一個遇到顏色為 $C$ 的點是哪個。
\end{frame}

\begin{frame}[fragile]{\btitle{子任務}}
	\begin{enumerate}
		\item $N, Q \le 2000$
		\item $U_i = i, V_i = i + 1$
		\item $C_i \in \{1, 2\}$
		\item 無額外限制
	\end{enumerate}
\end{frame}

\begin{frame}[fragile]{\btitle{$N, Q \le 2000$}}

	暴力

\end{frame}

\begin{frame}[fragile]{\btitle{$U_i = i, V_i = i + 1$}}
	先對每個顏色維護好有哪些點是這個顏色。
	
	由於圖是一條練,如果 $A \le B$ 我們可以用 lower bound 來找到第一個顏色為 $C$ 且大於等於 $A$ 的點就好。 

	$A > B$ 同理

\end{frame}

\begin{frame}[fragile]{\btitle{$C_i \in \{1, 2\}$}}
	由於詢問沒有修改,因此可以離線來做,我們先處理完顏色為 $1$ 的再處理顏色為 $2$ 的。
	\pause

	對於目前要處理的顏色,相同顏色的點我們將那個點權設為 $1$,不同顏色的點權設為 $0$,如此一來可以用一條路徑的總和來看這條路徑上是否有出現相同顏色的點。
	\pause

	我們將 $A$ 到 $B$ 拆成 $A$ 到 $l$ 和 $l$ 到 $B$,其中 $l$ 是 $A$ 跟 $B$ 的LCA,如果 $A$ 到 $l$ 有顏色 $C$,那我們找第一個出現的就好,如果沒有,那我們再到 $B$ 到 $l$ 上找最後一個出現顏色為 $C$ 的點就好,而這些東西可以用倍增之類的方式來維護。
\end{frame}

\begin{frame}[fragile]{\btitle{無額外限制}}
	延續 subtask 3,也是離線並且相同顏色的詢問一起處理,由於顏色很多,不可能每種顏色都建一個倍增表,因此需要一個能夠快速算出一條路徑上有多少東西並修改的資料結構。
	\pause

	輕重鍊剖分完全符合這個條件。但輕重鍊剖分又大又難寫,換個經典的替代方法,樹差分。
	\pause
	
	可以紀錄每個點 dfs 進入跟離開的時間 $in_i$ $out_i$ 做樹壓平,這樣有個性質是對於 $i$ 的子樹節點都會包含在 $in_i$ $out_i$ 之間,因此只要那個點要 + 1,我們可以在 $in_i$ 上 + 1, $out_i$ 上 - 1,如此一來,某個點 $i$ 到根節點上的路徑總和就可以用 $1$ 到 $in_i$ 的總和求得,因為需要修改,所以可以用二元索引樹(BIT)來維護。

	如此一來我們就可以用 subtask 3 的方式求出答案了。
\end{frame}

\end{document}
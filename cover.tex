\documentclass[a4paper]{article}
\title{Cover}
\usepackage[left=2cm,right=2cm,top=3cm,bottom=2cm]{geometry}
\usepackage{fontspec}
\usepackage{xeCJK}
\setCJKmainfont[]{Noto Sans Mono CJK TC}
\setCJKmonofont[]{Noto Sans Mono CJK TC}
\XeTeXlinebreaklocale "zh"
\XeTeXlinebreakskip = 0pt plus 1pt
\usepackage{anyfontsize}
\usepackage{t1enc}
\usepackage{listings}
\lstset{language=C++,basicstyle=\ttfamily}
\begin{document}

\begin{center}
\textbf{\huge TODO競賽名稱}\\
\vspace{5mm}
\textbf{\huge TODO競賽名稱\ 試題本}\\
\vspace{10mm}
\rule{17cm}{2pt}\\
\vspace{5mm}

\huge 競賽規則\\
\end{center}

\fontsize{14pt}{20pt}\selectfont
\begin{enumerate}
    \setlength\itemsep{0.5pt}
    \item 競賽時間:202TODO/TODO/TODO TODO:TODO \textasciitilde \, TODO:TODO,共 TODO 小時。
    \item 本次競賽試題共 TODO 題,每題皆有子任務。
    \item 為了愛護地球,本次競賽題本僅提供電子檔,不提供紙本。
    \item 每題的分數為該題所有子任務得分數加總;單筆子任務得分數為各筆繳交在該筆得到的最大分數。
    \item 本次初選比照南區賽提供記分板,複選比照全國賽不提供記分板。
    \item 全部題目的輸入皆為標準輸入。
    \item 全部題目的輸出皆為標準輸出。
    \item 所有輸入輸出請嚴格遵守題目要求,多或少的換行及空格皆有可能造成裁判系統判斷為答案錯誤。
    \item 每題每次上傳間隔為 120 秒,裁判得視情況調整。
    \item 所有試題相關問題請於競賽系統中提問,題目相關公告也會公告於競賽系統,請密切注意。
    \item 如有電腦問題,請舉手向監考人員反映。
    \item 如有如廁需求,須經過監考人員同意方可離場。
    \item 不得攜帶任何參考資料,但競賽系統上的參考資料可自行閱讀。
    \item 不得自行攜帶隨身碟,如需備份資料,請將資料儲存於電腦 D 槽。
    \item 競賽中請勿交談。請勿做出任何會干擾競賽的行為。
    \item 如需使用 C++ 的 \lstinline{std::cin} 或 \lstinline{std::cout} 可將以下程式碼插入 main function 以及將 \lstinline{endl} 取代為 \lstinline{'\n'} 來優化輸入輸出速度。唯須注意不可與 \lstinline{cstdio} 混用。
        \begin{lstlisting}
        std::ios::sync_with_stdio(false);
        std::cin.tie(nullptr);
        \end{lstlisting}

\end{enumerate}

\begin{center}
\rule{17cm}{2pt}\\
\end{center}
\end{document}
